%% Submissions for peer-review must enable line-numbering 
%% using the lineno option in the \documentclass command.
%%
%% Preprints and camera-ready submissions do not need 
%% line numbers, and should have this option removed.
%%
%% Please note that the line numbering option requires
%% version 1.1 or newer of the wlpeerj.cls file, and
%% the corresponding author info requires v1.2

\documentclass[fleqn,10pt,lineno]{wlpeerj} % for journal submissions
% \documentclass[fleqn,10pt]{wlpeerj} % for preprint submissions

\title{The recent expansion of Fox Sparrow (\textit{Passerella iliaca iliaca}) breeding range into the northeastern United States}

\author[1]{John D. Lloyd}
\affil[1]{Vermont Center for Ecostudies, Norwich, VT, USA}
\corrauthor[1]{John D. Lloyd}{jlloyd@vtecostudies.org}

% \keywords{Fox Sparrow, \textit{Passerella iliaca}, distribution, biogeography, Maine, New Hampshire, range expansion}

\begin{abstract}
The breeding range of the Red Fox Sparrow (\textit{Passerella iliaca iliaca}) is generally recognized as comprising the boreal forest of Canada. However, recent observations suggest that the species is present during the summer months throughout much of the northeastern U.S., unexpected for a species characterized as a passage migrant in the region. To clarify, I conducted a literature review to document the historical status of the species in the northeastern U.S. and then analyzed observations submitted to eBird to describe its recent and current status in the region. Historical accounts consistently identify Fox Sparrow as a passage migrant through the region during early spring and late fall. Beginning in the early 1980s, observers began noting regular extralimital records of Fox Sparrow in northern Maine. A single nest was discovered in the state in 1983, and another in northern New Hampshire in 1997. Despite the paucity of breeding records, observations submitted to eBird suggest that the southern limit of the breeding range of Fox Sparrow has expanded rapidly to the south and west in recent years. The proportion of complete checklists submitted to eBird that contained at least one observation of Fox Sparrow grew at an annual rate of 18\% from 2003-2016 and was independent of observer effort. Fox Sparrow now occurs regularly on mountaintops and in young stands of spruce (\textit{Picea} spp.) and balsam fir (\textit{Abies balsamea}) during the breeding season throughout northern and western Maine and northern New Hampshire, with occasional records from the Green Mountains of Vermont and the Adirondack Mountains of New York. The cause of this rapid expansion of its breeding range is unknown, but may be related to an increase in the amount of young conifer forest in the northeastern U.S. created by commercial timber harvest.
\end{abstract}

\begin{document}

\flushbottom
\maketitle
\thispagestyle{empty}

\section*{Introduction}
Fox Sparrow (\textit{Passerella iliaca}) is a polytypic species that breeds in montane and boreal forest across western and northern North America. Red Fox Sparrow, the nominate subspecies (\textit{P}. \textit{i}. \textit{iliaca}), nests in early-successional coniferous or mixed forests; shrubby thickets along waterways and wetlands; and stunted conifer forests on mountaintops or cool coastlines from Manitoba eastward to the Maritime Provinces of Canada \citep{Bisson1996-mp,McLaren2007-go,Stewart2015-id,Artuso2018-kd}. Most general references identify the subspecies’ breeding range as extending into the eastern United States only in the northernmost part of Maine \citep{Rising1996-sy,Sibley2000-rs,Weckstein2002-px}. However, anecdotal reports from birders and observations submitted to eBird (http://www.ebird.org), a web platform that documents bird distributions using observations submitted by citizen-scientists, suggest that Fox Sparrow now occurs regularly during the breeding season throughout northern and western Maine and as far south and west as the mountains of central and northern New Hampshire. This would constitute a fairly rapid southward expansion of the species’ breeding range, on the order of several hundred kilometers, yet this phenomenon has remained undescribed in the ornithological literature. Here, in an effort to address this gap and to clarify the status of Fox Sparrow as a breeding species in the northeastern U.S., I review historical and current literature describing the distribution of Fox Sparrow in this region and describe temporal changes in occurrence using data submitted to eBird by citizen scientists. 

\section*{Methods}
I began by reviewing general summaries of bird distribution for the two northeastern states that appear to represent the leading edge of the southward expansion of Fox Sparrow breeding range: Maine, the bird life of which still has only a single definitive reference \citep{Palmer1949-ig}, and New Hampshire, which has both an historical account of bird distributions \citep{Allen1903-xq} and an authoritative recent update \citep{Keith2013-gt}. I also consulted two older accounts of the birdlife of New England \citep{Samuels1875-jo,Forbush1929-pq}. These accounts formed the basis for understanding the historic distribution of Fox Sparrow in Maine and New Hampshire. 

I supplemented these historical accounts, and located more recent references to the species’ distribution, by searching Google Scholar with the string “("fox sparrow" OR "passerella iliaca") AND ("New Hampshire" OR Maine)”. I also manually searched for records of Fox Sparrow in regional reports for the northeastern U.S. and southeastern Canada that appeared in American Birds, National Audubon Society Field Notes, and North American Birds from 1980-2006. Finally, I searched breeding-bird atlases for Maine, New Hampshire, Vermont, and New York for any information pertaining to the presence of Fox Sparrows.

To describe and quantify recent changes in the incidence of Fox Sparrow reports during the breeding season, I used data submitted to eBird. To describe the current distribution, I used the July 2018 version of the eBird Basic Dataset, which includes both incidental (i.e., potentially incomplete checklists without associated information on observer effort) and complete-checklist records (i.e., checklists that include all species observed and that have an accompanying measure of observer effort). 

To quantify temporal changes in the incidence of Fox Sparrow observations, I began by downloading the 2016 version of the eBird Reference Dataset \citep{Fink2017-om}, which includes only complete checklists and is zero-filled such that Fox Sparrows were treated as absent from any checklist without a positive record of the species. I then filtered this dataset to include only checklists from June and July, which should eliminate nearly all records of migrant birds, as Fox Sparrow migration through the region occurs very early in April \citep{Weckstein2002-px}. The mean arrival date on the breeding grounds in Newfoundland, Canada, for example, was 9 April over a six-year period from 1973-1978 \citep{Threlfall1982-wd}. Regionally, the species is also a late fall migrant, with southward movements typically not beginning until mid- to late September \citep{Weckstein2002-px}. I further reduced the dataset by including only checklists from counties in the northern halves of Maine (Aroostook, Penobscot, Piscataquis, Somerset, Franklin, and Oxford) and New Hampshire (Coos and Grafton). I limited the analysis to this extent because it included nearly all of the potentially suitable habitat in both states, namely stunted krummholz forest on high mountaintops and young stands of balsam fir (\textit{Abies balsamea}) and spruce (\textit{Picea} spp.) regenerating after harvest in the region’s extensive commercial forestland. I estimated the elevation for each complete checklist containing an observation of Fox Sparrows using a 30-m resolution digital-elevation model \citep{farr2007shuttle}.

Analyzing the regional occurrence of Fox Sparrows in eBird checklists posed two problems. First, because of the species’ rarity and its localized distribution in the region, many checklists were submitted from the same locale on approximately the same date and presumably recorded the presence of the same individual or individuals. For example, when an individual was located on an accessible mountaintop, many birders would actively seek out that individual and submit a checklist recording its presence. As these are not independent observations, this phenomenon would potentially lead to an overestimate of the frequency with which the species is observed in the region. At the same time, the localized regional distribution of Fox Sparrows resulted in an exceedingly small proportion of checklists reporting the species. 

To address both issues, I used as my dependent variable not the raw proportion of checklists with Fox Sparrow but instead the proportion of 10 km$^{2}$ grid cells within the region that had at least one complete checklist containing Fox Sparrow. To accomplish this, I first overlaid a 10 km$^{2}$ grid atop the study area within the QGIS version 2.18 Geographic Information System \citep{QGIS_Development_Team2016-iw} and then determined whether each grid cell contained at least one complete checklist with a record of Fox Sparrow for each year. Doing so avoided the problem of treating multiple observations of the same bird as independent and also yielded clearer insight into large-scale changes in the occurrence of Fox Sparrow. The choice of the grid size was essentially subjective, but reflected a trade-off between cells that were too large to accurately quantify changes in distribution over time and cells that were too small to accurately record the location where Fox Sparrows were detected during traveling counts submitted to eBird (i.e., situations where an observer started observing birds in one grid cell, which would be identified as the location of the observation in eBird, but walked far enough to cross into an adjacent grid cell). 

I analyzed changes in the proportion of grid cells occupied by Fox Sparrow over time using a generalized linear model, with the conditional distribution of the response variable assumed to be binomial (i.e., a grid cell did or did not contain a record of Fox Sparrow in that year). I included year as the predictor variable in the model. Any change in the proportion of cells with Fox Sparrow records could be due in part to increased observer effort over time, so I analyzed a second model that included both year and a measure of observer effort, which I quantified as the summed value of observation hours spent on each complete checklist in the grid cell. I used analysis of deviance to choose the preferred model for inference. I tested for overdispersion via a chi-squared test of the ratio of the squared sum of Pearson residuals to the residual degrees of freedom \citep{Venables2002-gt}. None of the models showed evidence of overdispersion (i.e., $\chi^2$ P ${<}$ 0.05), and as such I made no adjustments. 

Although the measure of observer hours spent birding can help address the confounding effect of effort on the apparent frequency with which a rare species like Fox Sparrow is detected, it does not address changes in observer behavior that may increase the likelihood of encountering the species. For example, over time, birders may have become more adventurous in their explorations and more likely to visit the locales - remote mountaintops or commercial forestlands with limited access - potentially inhabited by Fox Sparrows. To address this potential source of confounding, I repeated the above analysis for two other species that occupy similar habitat as Fox Sparrows: Bicknell’s Thrush (\textit{Catharus bicknelli}) and Blackpoll Warbler (\textit{Setophaga striata}). Bicknell’s Thrush populations have likely been declining slowly in the region \citep{Lambert2008-jj,King2008-qx} in recent decades, whereas Blackpoll Warblers have shown no significant trend in numbers \citep{King2008-qx}, so any increase in the proportion of grid cells occupied by these species in the study area might reflect increased observer activity in habitat suitable for Fox Sparrows. 

All analyses were conducted in R version 3.4.4 \citep{R_Core_Team2018-cc}. 

\section*{Results}
\subsection*{Historical distribution of Fox Sparrows in the northeastern United States}
Early reviews of the region’s avifauna were unequivocal in describing Fox Sparrow as a passage migrant. \cite{Palmer1949-ig} identified the species as “transient” in Maine; \cite{Allen1903-xq} as a “rather common migrant” in New Hampshire, a conclusion also reached by \cite{Keith2013-gt} in their comprehensive review of historical and recent records; and both \cite{Samuels1875-jo} and \cite{Forbush1929-pq} classified the species as a spring and fall migrant throughout New England. None of these references indicated that Fox Sparrows were present in the region outside of early spring and late fall. 

\subsection*{Recent distribution of Fox Sparrows in the northeastern United States}
Several checklists submitted as historical records to eBird reported Fox Sparrows in far northern Maine near Madawaska during the breeding season (47.36$^{\circ}$N, -68.33$^{\circ}$W) as early as 1978. The first strong indication that Fox Sparrows were breeding in the northeastern U.S. came in 1981, documented in an eBird checklist submitted as an historical record in 2017. Three observers (Jeff Cherry, Jim Eckler, and Lynn Sheldon), working on a study of the effects on birds of spraying pesticides to control an outbreak of eastern spruce budworm (\textit{Choristoneura fumiferana}), located a singing Fox Sparrow in northern Somerset County, Maine (46.36$^{\circ}$N, -70.06$^{\circ}$W) on several occasions between 21 June and 3 July. As described in the comments associated with the checklist, this bird was reported to have responded aggressively to a broadcast recording of a Fox Sparrow song. Confirmation of nesting was obtained further north (47.35$^{\circ}$N, -68.28$^{\circ}$W), several kilometers south of the U.S.-Canada border in Aroostook County, Maine, in 1983 by Peter Vickery during Maine’s first breeding bird atlas \citep{Adamus1988-xw}. 

Outside of these scattered records, evidence of a widespread southward range expansion first began emerging in the mid-1980s in southern Quebec. In 1985 and again in 1986, birders observed singing Fox Sparrows “well south” of their breeding range in southeast Quebec near the border of Maine \citep{Yank1986-pj,Yank1985-pj}. In 1988, four Fox Sparrows in two different locations in southern Quebec were also noteworthy for being “well s. of their usual summer range” \citep{Gosselin1988-vk}. In 1993, three singing Fox Sparrows were observed in “suitable nesting habitat in n. Maine during late June” \citep{Petersen1993-kb}. The species was still uncommon enough to warrant attention in 1999 and 2000, when notable records of singing birds were obtained in central Maine near Mount Katahdin \citep{Petersen1999-kb}. Meanwhile, in northern New Hampshire, a Fox Sparrow was discovered singing in 1996, with nesting confirmed at the same location (45.25$^{\circ}$N, -71.21$^{\circ}$W) in 1997 \citep{Keith2013-gt}. This was the first confirmed nesting attempt by Fox Sparrow in New Hampshire, none having been found during the state’s breeding bird atlas, conducted from 1981-1986 \citep{Foss1994-nl}. Fox Sparrows were also absent from statewide breeding-bird atlases conducted in New York from 1980-1985 \citep{Andrle1988-jk} and 2000-2005 \citep{McGowan2008-em}, and in Vermont from 1976-1981 \citep{Laughlin1985-mn} and 2003-2007 \citep{Renfrew2013-kx}.

Based on all eBird records submitted through July 2018, including incidental observations or otherwise incomplete checklists, the current distribution of Fox Sparrow during June and July in the northeastern U.S. ranges from northern Maine, the mountains of central and western Maine, northern New Hampshire, the White Mountains of central New Hampshire, the central Green Mountains of Vermont, and into the High Peaks of the Adirondack Mountains of New York (Fig.\ref{fig:Figure1}). The species now occurs as far south as 44$^{\circ}$N in the high peaks of the White Mountains and as far west as -74$^{\circ}$W in the Adirondack Mountains, or roughly 3$^{\circ}$ further south and 6$^{\circ}$ further west than it did 30 years earlier.

The single record from southern Maine (44.74$^{\circ}$N, -68.73$^{\circ}$W), in July 2016) is of unknown significance, but its location, at a frequently birded nature preserve near the city of Bangor with no other breeding-season records of Fox Sparrow, suggests that bird observed was not on a breeding territory. The records in Vermont stem from surveys conducted by citizen-scientists working on the Vermont Center for Ecostudies’ Mountain Birdwatch Program, an annual, trail-based survey of birds breeding at high elevations in the northeastern U.S. The detections occurred on 1 survey route in 2011, 1 in 2012, and 3 different routes in 2016. The sole observation of Fox Sparrow in New York occurred on Whiteface Mountain in 2012, with multiple observers detecting at least one singing male during June and July. 

\begin{figure}[ht]\centering
\includegraphics[width=\linewidth]{Figure1}
\caption{Observations of Fox Sparrow during June and July in the northeastern United States reported to eBird as of July 2018.}
\label{fig:Figure1}
\end{figure}

\subsection*{Temporal changes in the frequency of Fox Sparrow detections in the northeastern United States}
The proportion of grid cells containing complete eBird checklists that noted the presence of Fox Sparrow increased from 2003-2016 ($b_{year}$ = 0.18, 95\% CI = 0.097 - 0.275; P < 0.001; Fig. \ref{fig:Figure2}), ranging from a low of 0\% in 2003 (0/28 cells), 2004 (0/29), 2006 (0/32), and 2007 (0/41) to a high of 8.7\% (20/231 grid cells) in 2015. Despite a large increase in observer effort – ranging from a low of 145.8 hours in 2005 (with a total of 20 grid cells containing at least 1 complete checklist) to a high of 3063.5 hours in 2016 (with a total of 277 grid cells containing at least 1 complete checklist) – the model containing a term for observer effort was not preferred relative to the simpler model containing only the effect of year (P = 0.67). Overall, the proportion of grid cells containing a complete checklist with a Fox Sparrow detection increased 18\% per year from 2003-2016. Checklists containing observations of Fox Sparrow in Maine tended to occur at lower elevations (median = 480 m, range = 40 – 1,592) than did checklists with observations in New Hampshire (median = 1,000 m, range = 360 – 1,888) (Fig. \ref{fig:Figure3}). 

\begin{figure}[ht]\centering
\includegraphics[width=\linewidth]{Figure2}
\caption{The proportion of 10 km$^{2}$ grid cells in northern and western Maine and northern New Hampshire containing breeding-season records of Fox Sparrows reported to eBird increased 18\% per year from 2003-2016 (top panel). Neither Bicknell’s Thrush (middle panel) nor Blackpoll Warbler (bottom panel), both of which occur in similar forest types as Fox Sparrow, showed a significant temporal trend in the proportion of grid cells reported as occupied.}
\label{fig:Figure2}
\end{figure}

\begin{figure}[ht]\centering
\includegraphics[width=\linewidth]{Figure3}
\caption{Fox Sparrow observations recorded in eBird tended to occur at lower elevations in Maine (A) than in New Hampshire (B).}
\label{fig:Figure3}
\end{figure}

Neither Blackpoll Warbler nor Bicknell’s Thrush showed a similar pattern of increasing frequency of detection (Fig. \ref{fig:Figure2}). The proportion of grid cells containing a record of Blackpoll Warbler during June and July may have declined slightly, although results were equivocal ($b_{year}$ = -0.030, 95\% CI = -0.064 - 0.005; P = 0.09), whereas the proportion of grid cells containing a record of Bicknell’s Thrush was apparently steady ($b_{year}$ = 0.03, 95\% CI = -0.029 - 0.109; P =0.30; Fig. \ref{fig:Figure2}). As with Fox Sparrow, the simpler model without an effect of observer effort was preferred for both species (Bicknell’s Thrush, P = 0.31; Blackpoll Warbler, P = 0.06). 

\section*{Discussion}
Red Fox Sparrows appear to be in the midst of a significant southward expansion of their breeding range. I found no evidence that the species was present in the northeastern U.S. during the breeding season until the late 1970s or early 1980s, yet they are now widely reported in New Hampshire and Maine during June and July. Observers have confirmed nesting in both states. This amounts to an approximately 400 km breeding-range extension during the span of approximately 30 years. 

Given that Fox Sparrows nest in remote or difficult-to-access locales — clearcuts and mountaintops — it is difficult to rule out the possibility that they may have nested in small numbers in the northeastern U.S. in the past. However, the avifauna of the region’s mountains has been well-described since the early 1900s, yet no breeding-season records for the species exist in the historical literature. Likewise, although the commercial forestlands of western and northern Maine are rarely visited by birders, numerous ornithological investigations conducted during the spruce-budworm epidemic of the 1980s failed to find the species, suggesting that it was far less common than at present. And, during the last decade, the increased frequency of breeding-season eBird reports of Fox Sparrow in Maine and New Hampshire appears to reflect a real increase in the species’ abundance and extent of occurrence, not an increase in observer effort in the right habitat; Bicknell’s Thrush and Blackpoll Warbler, which co-occur with Fox Sparrow during the breeding season, have shown no such increase.

During the mid-1980s, birders in southern Quebec began noting what were then considered extralimital records of Fox Sparrow, at about the same time that observers in far northwestern Maine first discovered evidence that Fox Sparrows were nesting in the state. The breeding range for Fox Sparrows continued to expand south and west out of southern Quebec and northern Maine over the next decade, reaching northern New Hampshire by the mid-1990s. Although Fox Sparrows have not spread substantially further west since then – only a handful of breeding-season records exist for Vermont, and only one for New York – they have continued moving south and now occur regularly during summer throughout the White Mountains of New Hampshire.

Fox Sparrows found in the northeastern U.S. during the breeding season tend to occur either in krummholz forest at high elevations or in young stands of spruce and fir regenerating after harvest, which is consistent with descriptions of nesting habitat occupied in the core of the species’ range in Canada \citep{Weckstein2002-px}. Other locales with potentially suitable nesting habitat include the Green Mountains of Vermont and the Adirondack Mountains of New York, both of which contain areas of krummholz forest. Extensive stands of young spruce-fir forest at lower elevations are uncommon outside of the commercial forestlands of western and northern Maine, although parts of far northeastern Vermont and northern New Hampshire support lowland spruce-fir forests that might provide suitable conditions for nesting Fox Sparrows following harvest.

Why the distribution of Fox Sparrows has expanded so rapidly is not clear. Breeding Bird Survey data reveal no evidence of population increases in eastern Canada \citep{Environment_and_Climate_Change_Canada2017-hp}, as might be expected if growth in the core of the range was forcing individuals to wander south in search of suitable breeding habitat. The arrival of Fox Sparrows in Maine coincides approximately with the extensive salvage logging that occurred during and after the last outbreak of eastern spruce budworm, which created vast areas of young spruce-fir forest in northern and western parts of the state. Given that Fox Sparrows will nest in these young stands, the species’ southward expansion may have been in part due to the increased availability of nesting habitat created by logging; \cite{Banks1970-rh} suggested a similar explanation for the spread of Fox Sparrows onto the western slopes of the Cascade Mountains of Oregon (but see \cite{Marshall2003-om}). 

\section*{Conclusions}
Red Fox Sparrows were historically a passage migrant through the northeastern U.S. during late spring and early fall, but within the past few decades have become widespread and regular during the breeding season. Although confirmed nesting records remain scarce, the consistent presence of singing males suggests that the southern limit of the breeding range now extends several hundred kilometers south of where it is placed on most published range maps. In particular, the species’ breeding range now encompasses the northwestern half of Maine and the highlands of northern New Hampshire from the White Mountains to the U.S.-Canada border. Scattered recent records in Vermont and New York, along with the presence of suitable habitat in the higher mountains of both states, suggests that Fox Sparrow may continue expanding its range to the west.

\section*{Acknowledgments}

Thanks to C.C. Rimmer and K.P. McFarland for constructive reviews of an early draft of this manuscript.

\bibliography{bibliography}

\end{document}